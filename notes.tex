\documentclass[twocolumn,draft]{article}

\usepackage{savetrees}

\usepackage[utf8]{inputenc}
\usepackage[T1]{fontenc}
\usepackage{microtype}

\usepackage{amssymb}
\usepackage{amsfonts}
\usepackage{amsmath}
\usepackage{amsthm}

\usepackage[english]{babel}

\RequirePackage[l2tabu, orthodox]{nag}
\usepackage[all,warning]{onlyamsmath}

\usepackage{hyperref}

\usepackage{cuted}

\usepackage[style=numeric-comp]{biblatex}
\bibliography{notes}

\title{Genome-wide scan for linear mixed models}
\author{Danilo Horta}
\date{\today}

\newcommand{\normal}[2]{\mathcal{N}\left(\, {#1} \,;\,~{#2} \,\right)}
\newcommand\Normal[3]{\normal{ {#1} \, \big| \, {#2} }{ {#3} }}

\newcommand{\texto}[1]{~~\text{#1}~~}

\newcommand{\til}[1]{\tilde{#1}}

\newcommand{\paren}[1]{\left(#1\right)}

\newcommand{\bfy}{\mathbf{y}}
\newcommand{\bfb}{\mathbf{b}}

\newcommand{\pinv}[1]{{#1}^{\dagger}}
\newcommand{\inv}[1]{{#1}^{-1}}
\newcommand{\trans}[1]{{#1}^\intercal}

\newcommand{\rmF}{\mathrm{F}}
\newcommand{\rmG}{\mathrm{G}}
\newcommand{\rmL}{\mathrm{L}}
\newcommand{\rmR}{\mathrm{R}}
\newcommand{\rmK}{\mathrm{K}}
\newcommand{\rmX}{\mathrm{X}}
\newcommand{\rmW}{\mathrm{W}}
\newcommand{\rmU}{\mathrm{U}}
\newcommand{\rmV}{\mathrm{V}}

\newcommand{\bfalpha}{\boldsymbol{\alpha}}
\newcommand{\bfbeta}{\boldsymbol{\beta}}

\newcommand{\haf}{\frac{1}{2}}


\begin{document}
	\maketitle

\section{Single block}

Let
\begin{equation*}
  \bfy \sim \normal{\rmF\bfalpha}{\rmK}
\end{equation*}
be the marginal likelihood of a LMM with a single covariate block $\rmF$.
The maximum likelihood estimation of the fixed effects is
\begin{equation*}
	\hat\bfalpha = \pinv{(\trans{\rmF}\pinv{\rmK}\rmF)}
	               \trans{\rmF}\pinv{\rmK}\bfy.
\end{equation*}
In practice, the method of least squares can be used to solve the above
equation without explicitly finding pseudoinverses.

\section{Double block}

Let
\begin{equation*}
  \bfy \sim \normal{\rmF\bfalpha + \rmG\bfbeta}{\rmK}
\end{equation*}
be the marginal likelihood of a LMM with two covariate blocks $\rmF$ and
$\rmG$.
We want to solve
\begin{equation*}
	\begin{bmatrix}
		\hat\bfalpha \\
		\hat\bfbeta
	\end{bmatrix} = \pinv{(\trans{\rmX}\rmX)}\trans{\rmX}\pinv{\rmK^{\haf}}\bfy
\end{equation*}
for
\begin{equation*}
\rmL = \pinv{\rmK^{\haf}}\rmF,~~ \rmR = \pinv{\rmK^{\haf}}\rmG\texto{and}
	\rmX =
		\begin{bmatrix}
			\rmL & \rmR
		\end{bmatrix}.
\end{equation*}

We have

\begin{strip}
	\begin{equation*}
		\pinv{(\trans{\rmX}\rmX)} =
		\begin{bmatrix}
			\pinv{(\trans{\rmL}\rmL)} + \pinv{(\trans{\rmL}\rmL)} (\trans{\rmL}\rmR)
				\pinv{\rmW} \trans{(\trans{\rmL}\rmR)} \pinv{(\trans{\rmL}\rmL)}
				& - \pinv{(\trans{\rmL}\rmL)} (\trans{\rmL}\rmR)
				\pinv{\rmW} \\
			- \pinv{\rmW} \trans{(\trans{\rmL}\rmR)}
			\pinv{(\trans{\rmL}\rmL)} & \pinv{\rmW}
		\end{bmatrix},
	\end{equation*}
\end{strip}
from Eq. 3 of~\cite{rohde1965generalized}, where
$\rmW = \trans{\rmR}\rmR - \trans{\rmR}\rmL\pinv{(\trans{\rmL}\rmL)}
\trans{\rmL}\rmR$.

Defining $\rmA = \trans{\rmL}\rmL$ and $\rmB = \trans{\rmL}\rmR$ leads us to
\begin{equation*}
	\pinv{(\trans{\rmX}\rmX)}\trans{\rmX} =
		\begin{bmatrix}
			\pinv{\rmA}\trans{\rmL}
			+ \pinv{\rmA}\rmB\pinv{\rmW}\trans{\rmB}\pinv{\rmA}\trans{\rmL}
			-\pinv{\rmA}\rmB\pinv{\rmW}\trans{\rmR}\\
			-\pinv{\rmW}\trans{\rmB}\pinv{\rmA}\trans{\rmL}
			+ \pinv{\rmW}\trans{\rmR}
		\end{bmatrix}.
\end{equation*}
Finally,
\begin{align*}
	&\pinv{(\trans{\rmX}\rmX)}\trans{\rmX} \pinv{\rmK^{\haf}}\bfy = \\
	&\begin{bmatrix}
		\pinv{\rmA}\trans{\rmF}\pinv{\rmK}\bfy
		+ \pinv{\rmA}\rmB\pinv{\rmW}\trans{\rmB}\pinv{\rmA}\trans{\rmF}\pinv{\rmK}\bfy
		-\pinv{\rmA}\rmB\pinv{\rmW}\trans{\rmG}\pinv{\rmK}\bfy \\
		\pinv{\rmW}\trans{\rmG}\pinv{\rmK}\bfy
		-\pinv{\rmW}\trans{\rmB}\pinv{\rmA}\trans{\rmF}\pinv{\rmK}\bfy
	\end{bmatrix}.
\end{align*}
A robust implementation of the above equation has to:
(i) associate matrix multiplications in such a way that a sequence of
$\pinv{\rmK}\dots\pinv{\rmK}$ is avoided;
and (ii) handle low-rank matrices $\rmW$.
A better association of matrix multiplications is given by
\begin{align*}
	&\pinv{(\trans{\rmX}\rmX)}\trans{\rmX} \pinv{\rmK^{\haf}}\bfy = \\
	&\begin{bmatrix}
		\pinv{\rmA}\trans{\rmF}\pinv{\rmK}\bfy
		+ \pinv{\rmA}\rmB\pinv{\rmW}(\trans{\rmB}\pinv{\rmA}\trans{\rmF}\pinv{\rmK}
		\bfy-\trans{\rmG}\pinv{\rmK}\bfy) \\
		\pinv{\rmW}(\trans{\rmG}\pinv{\rmK}\bfy
		-\trans{\rmB}\pinv{\rmA}\trans{\rmF}\pinv{\rmK}\bfy)
	\end{bmatrix}.
\end{align*}
$\pinv{\rmW}$ can be found via economic SVD decomposition.

\section{Batch scan}

Given a $n\times p$ covariate block $\rmG$, we want to quickly infer the
maximum likelihood estimations of $\bfalpha_i$ and $\beta_i$, for
$i \in \{1, \dots, p\}$ denoting the $\rmG$ columns.
We define a diagonal matrix
\begin{equation*}
	\rmW = \dotd{\trans{\rmG}}{\pinv{\rmK}\rmG} -
	       \dotd{\trans{\rmG}}{\pinv{\rmK}\rmF
				 \pinv{(\trans{\rmL}\rmL)}\trans{\rmF}\pinv{\rmK}\rmG},
\end{equation*}
where $\dotd{\cdot}{\cdot}$ is a function that returns the diagonal elements
of a matrix multiplication with asymptotically lower computational cost and
memory use.

Clearly,
\begin{equation*}
	\hat\bfbeta =
	\begin{bmatrix}
		\hat\beta_1 \\
		\hat\beta_2 \\
		\vdots \\
		\hat\beta_p
	\end{bmatrix}
	= \pinv{\rmW}(\trans{\rmG}\pinv{\rmK}\bfy
	-\trans{\rmB}\pinv{\rmA}\trans{\rmF}\pinv{\rmK}\bfy).
\end{equation*}
We know that
\begin{equation*}
	\hat\bfalpha_i = \pinv{\rmA}\trans{\rmF}\pinv{\rmK}
	(\bfy + (\rmG_i \rmW_i \trans{\rmG_i})
	(\pinv{\rmK}\rmF\pinv{\rmA}\trans{\rmF}\pinv{\rmK}\bfy - \pinv{\rmK}\bfy)),
\end{equation*}
where $\rmG_i$ is the $i$-th column of $\rmG$.
For one-shot computation we do
\begin{equation*}
	\hat\bfalpha = \pinv{\rmA}\trans{\rmF}\pinv{\rmK}
	\paren{\bfy \oplus \rmG_i \otimes
	\trans{\paren{(\rmW_i \trans{\rmG_i})
	  (\pinv{\rmK}\rmF\pinv{\rmA}\trans{\rmF}\pinv{\rmK}\bfy
		 - \pinv{\rmK}\bfy)}}},
\end{equation*}
where $\oplus$ and $\otimes$ are element-wise summation and multiplication
with broadcasting.


\printbibliography\

\end{document}
